
	Del desarrollo de este trabajo se puede concluir que la estimación de los errores es de suma importancia, porque sin la correción realizada se hubiese confundido la llegada al punto \texttt{B} cuando en realidad se el vehículo en el punto \texttt{C}.\\

	En cuanto a las limitaciones dadas por el poder de cómputo, se garantiza que la cota dada por le incerteza en $x$ se cumple. El problema de la incertidumbre en $y$ se supone que proviene de la proximidad entre los puntos \texttt{A,B} y \texttt{C} en dicho eje. Por lo tanto, para garantizar mejor resolución, se requieren mayor cantidad de muestras.\\

	Se debe tener en cuenta también, que para realizar el cálculo del valor mínimo de $N$ se tomaron muchas cotas máximas. Por lo tanto, dicho valor puede ser mucho menor. Sin embargo, como los errores se deben al instrumental utilizado, no se puede aplicar esta técnica a cualquier problema de trayectoria de un vehículo (los errores pueden diferir dependiendo de la instrumentación).


