
	Para realizar el cálculo de la trayectoria (posición del vehículo a lo largo del tiempo) se debe integrar dos veces la aceleración. Sin embargo, como la medición difiere del valor verdadero, al integrarse propaga los errores alterando el valor de la trayectoria.
	\begin{align*}
		{a}_{med}(t) &= {a}_{real}(t) + {\textit{errores}}\,(t)\\
		v_{calc}(t) = v_{inicial} + \int^t_{t_0} a_{med}(\tau)\: d\tau &= v_{real}(t) + \int^t_{t_0} \textit{errores}\,(\tau) d\tau \\
		p_{calc}(t) = p_{inicial} + \int^t_{t_0} v_{calc}(\tau)\: d\tau &= p_{real}(t) + \int^t_{t_0}\left[\int^\tau_{t_0} \textit{errores}\,(\nu) d\nu\right] d\tau %\\
%		\Rightarrow p_{real} = p_{inicial} + \int^t_{t_0} \left[\int^\tau_{t_0} a_{med}\,(\nu) d\nu\right]&+v_{inicial}\; d\tau - \int^t_{t_0}\left[\int^\tau_{t_0} \textit{errores}\,(\nu) d\nu\right] d\tau 
	\end{align*}

	Sin embargo, como toda medición, la trayectoria resultante tendrá incerteza. El error de posición debe ser la mitad de la distancia mínima entre las mediciones. A continuación se desarrolla la relación de dicho error $\delta_p$ con las muestras.
	\begin{equation*}
		\delta_p = \left\|\int^{t_f}_{t_0}\int^{t}_{t_0} (\hat{b}-{b}) + (\hat{s} - s) \cdot a_{med} + \zeta \, dt^{'}\right\| \leq \frac{t^2}{2} |\hat{b}-b| + \underbrace{\int^{tf}_{t_0}\int^t_{t_0}(\hat{s}-s)\cdot a_{med}(t) \, dt}_{\leq\frac{t^2}{2}\cdot|a_{\max}|\cdot|\hat{s}-s|} + \int^{tf}_{t_0}\int^t_{t_0}\zeta(t) \, dt
	\end{equation*}
	Dicha desigualdad puede ser acotada por cotas de sus términos. Los términos correspondientes a la diferencia entre los estimadores y los parámetros reales pueden acotarse suponiendo que las estimaciones son con un $N$ suficientemente grande para que su distribución de probabilidad sea normal. Con dicha hipótesis, se puede afirmar que la mayoría ($\approx 99\%$) de los valores estimados se encuentran a 3 desvios de distancia de la media. Dado que la media de los estimadores es el parámetro a estimar (estimador insesgado), entonces $|\hat{\gamma} - \gamma|\leq 3\,\sigma_\gamma$ donde $\gamma$ es cualquiera de los parámetros estimados.Por otro lado, se puede acotar la integral del ruido por $3\,\sigma_\eta$ debido a que se trata de un proceso de Wiener o de \emph{random walk}. En consecuencia:
	\begin{equation*}
		\delta_p < \frac{t^2_f}{2} (3\sigma_b  + 3|a_{\max}|\sigma_s) + 3\sigma_\eta %< \frac{\min\{\textit{distancia}\}}{2}
	\end{equation*}

	\graficarEPS{0.6}{graf_trayectoria}{Trayectorias con y sin correción, con los posibles puntos de llegada.}{fig:tray}
	A partir de todo lo desarrollado en esta sección, se grafican las trayectorias corregida (y su incerteza) y sin correción en la Figura \ref{fig:tray}. Se puede ver allí, que el entorno de incertidumbre es muy pequeño resultando en que el vehículo no llegó a ninguno, aunque pasó por el \texttt{A}. Éste resultado es extraño y puede deberse a un mal cálculo de los $\delta_p$, porque por inspección se podría afirmar que el vehículo debería haber llegado al punto \texttt{C}.

