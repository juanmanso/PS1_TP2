\documentclass[10pt,a4paper]{article}

\usepackage[utf8]{inputenc}		% Configuro la codificación
\input{config.tex}				% Archivo con los comandos globales como Título y autores
\input{preamble.tex}
\input{aux_functions.tex}		% Se proveen un conjunto de funciones extras

% Defino el path de los includegraphics
\graphicspath{{./Figuras/}}		% Directorio que contiene los graficos

% Defino el path para los input de .tex y de .eps
\makeatletter
\def\input@path{{./Figuras/}{./Secciones/}{./Cover_page/}}
\makeatother

% Defino el path del listings
\ifListings
%% Cambiar el nombre de la carpeta si se utilizan Listings
	\lstinputpath{{../Octave/}}
\fi

\definecolor{myred}{rgb}{0.5,0,0}
\definecolor{mygreen}{rgb}{0,0.5,0}



\begin{document}
		% Carátula (formal o simple,_formal o _simple respectivamente) con Resumen
		% incluido e Índice (si es necesario configurar en config.tex) del informe
		\input{cover_formal.tex}
	\setcounter{page}{1}

	\section{Introducción}\label{sec:intro}
		
	A partir de la información provista por dos acelerómetros colocados en un vehículo, se desea obtener la trayectoria del mismo. Como dichas mediciones se ven afectadas por errores, se estimará dicha trayectoria sabiendo la naturaleza de los errores. 
	\begin{align}
		a_{medida} &= a_{real} + \mathit{errores} \\
		a_{medida} &= a_{real} + b + s \cdot a_{medida} + \zeta
		\label{eq:acel_teo}
	\end{align}

	A través de un ensayo previo, se determina que la relación entre la aceleración medida de la real está dada por la ecuación \eqref{eq:acel_teo} donde $b$ es el sesgo, $s$ el factor de escala y $\zeta$ es \emph{AWGN} de media nula y variancia $\sigma_\zeta$ conocida.

		
	\section{Modelo para la estimación}\label{sec:modelo}
		
	\graficarEPS{1.0}{geometria}{Geometría del probelma.}{fig:geom}

	Como se ve en la Figura \ref{fig:geom}, la aceleración del vehículo es $\underline{a}=[a_x \;\; a_y] = [-g\cdot \sen(\theta) \;\; -g\cdot \cos(\theta)]$. Por lo tanto, a través de la Ecuación \eqref{eq:acel_teo} se obtiene:
	\begin{align*}
		a_{med_x}(\theta_i) &= - \sen(\theta_i) \cdot g + b_x - s_x\cdot \sen(\theta_i)\cdot g + \zeta_x  \\
		a_{med_y}(\theta_i) &= - \cos(\theta_i) \cdot g + b_y - s_y\cdot \cos(\theta_i)\cdot g + \zeta_y 
	\end{align*}

	En el ensayo, se conoce la aceleración provista por los acelerómetros, como también el ángulo $\theta$. Así las incógnitas a estimar son $s_x$, $s_y$, $b_x$, $b_y$. Por lo tanto se hace el siguiente reordenamiento:
	\begin{align*}
		b_x  + s_x \cdot (-g\sen(\theta_i)) + \zeta_x &= a_{medida_x} + g\sen(\theta_i)\\
		b_y  + s_y \cdot (-g\cos(\theta_i)) + \zeta_y &= a_{medida_y} + g\cos(\theta_i)
	\end{align*}
	Debe notarse que si no hubiese rotación, no se podría saber si el error de medición se debe al error de escala (función de $\theta$) o el sesgo (término independiente). Al haber variación de $\theta$ en segundo ensayo, la distinción es posible. \\
	\indent A continuación (Ecuación \eqref{eq:modelo_i}) se presenta la expresión del modelo para la estimación de dichos parámetros\footnote{Para diferenciar las variables $X$ e $Y$ del modelo con las variables $x$ e $y$ espaciales, se representan con mayúsculas y minúsculas respectivamente.}:
	\begin{equation}
		Y(\theta_i) = \underbrace{\begin{bmatrix} a_{med_x}(\theta_i) +g\sen(\theta_i) \\[0.3em] a_{med_y}(\theta_i)+g\cos(\theta_i) \end{bmatrix}}_{Y_i} = \underbrace{\begin{bmatrix} 1 & -g\sen(\theta_i) & 0 & 0 \\[0.3em] 0&0& 1& -g\cos(\theta_i) \end{bmatrix}}_{A^i} \cdot \underbrace{\begin{bmatrix} b_x\\[0.3em] s_x\\[0.3em] b_y\\[0.3em] s_y \end{bmatrix}}_{X} + \underbrace{\begin{bmatrix} \zeta_x \\[0.3em] \zeta_y \end{bmatrix}}_{\eta}
		\label{eq:modelo_i}
	\end{equation}

	Matricialmente se obtiene por lo tanto:
	\begin{align*}
		\underbrace{\begin{bmatrix} Y_1 \\[0.3em] Y_2 \\[0.3em] \vdots\end{bmatrix}}_{Y} &= \underbrace{\begin{bmatrix} A^1\\[0.3em] A^2\\[0.3em] \vdots\end{bmatrix}}_{A}\cdot X + \eta\\
		\Rightarrow Y &= A\cdot X + \eta
	\end{align*}

	%% Otra forma, probar
	Otra forma de modelizar este problema es separando en $x$ e $y$ \vspace{0.1cm}. Se pueden resolver los 2 problemas por separado por la geometría del mismo. Rescribiendo $A^i = \begin{bmatrix} A^i_1 & \underline{0} \\[0.3em] \underline{0} & A^i_2 \end{bmatrix}$ y también $Y^i = \begin{bmatrix} Y^i_x \\[0.3em] Y^i_y \end{bmatrix}$ el modelo es:
	\begin{align*}
	\underbrace{\begin{bmatrix} A_1 & 0 \\[0.3em] 0 & A_2 \end{bmatrix}}_A \; X + \underbrace{\begin{bmatrix}\zeta^1_x\\ \vdots\\ \zeta^{n/2}_x \\ \zeta^1_y \\ \vdots \\ \zeta^{n/2}_y \end{bmatrix}}_\eta &= \underbrace{\begin{bmatrix} \underline{Y^T_x} \\[0.3em] \underline{Y^T_y} \end{bmatrix}}_Y
	\end{align*}



	\section{Estimación de los valores de los sesgos y factores de escala}\label{sec:estim}
                
	El estimador $\hat{X}$ es aquel que minimice $\mathbb{E}\left[||Y-\hat{Y}||^2\right]$, con $\hat{Y}=A\hat{X}$. Entonces:
	\begin{align*}
		< Y - A\hat{X}, A X> &= 0\\
		\mathbb{E}\left[ (A \hat{X}-Y)^T A X\right] &= 0\\
		\Rightarrow A^T A \hat{X} &= A^T Y
	\end{align*}

	Suponiendo que existe $\left(A^T A\right)^{-1}$, aplicado en el modelo donde se separan las variables $x$ e $y$:
	\begin{align*}
		\left(A^T A\right)^{-1} A^T &=\left(\begin{bmatrix} A^T_1 & 0\\0 & A^T_2\end{bmatrix} \begin{bmatrix} A_1 & 0\\0& A_2\end{bmatrix}\right)^{-1} A^T \\ &= \begin{bmatrix} (A^T_1 A_1)^{-1} & 0 \\ 0& (A^T_2 A_2)^{-1}\end{bmatrix} \begin{bmatrix} A^T_1&0\\0&A^T_2\end{bmatrix} \\&= \begin{bmatrix} (A^T_1 A_1)^{-1}A^T_1 & 0 \\ 0& (A^T_2 A_2)^{-1}A^T_2\end{bmatrix}\\
		\Rightarrow \hat{X} = \begin{bmatrix} \hat{X}_x \\ \hat{X}_y \end{bmatrix} &= \begin{bmatrix} (A^T_1 A_1)^{-1}A^T_1 & 0 \\ 0& (A^T_2 A_2)^{-1}A^T_2\end{bmatrix} \begin{bmatrix} Y_x\\Y_y \end{bmatrix}
	\end{align*}

	En consecuencia, se pueden realizar los dos problemas por separado como se afirmó en un principio. De aquí en adelante se harán los desarrollos para uno de ellos dado que el del otro será el mismo. \\ \indent Para calcular la covarianza de $\hat{X}$ es conveniente analizar si el estimador es insesgado, es decir si la esperanza del parámetro coincide con la del estimador.
	\begin{align*}
		\mathbb{E}[\hat{X}]&= \mathbb{E}\left[(A^T A)^{-1} A^T Y\right] \\&= (A^T A)^{-1} A^T \mathbb{E}[Y]\\
				&=(A^T A)^{-1} A^T \mathbb{E}[AX + \eta]\\ 
				&=(A^T A)^{-1} A^T (\mathbb{E}[AX] + \underbrace{\mathbb{E}[\eta]}_{=0})\\ 
				&=\underbrace{(A^T A)^{-1} A^T A}_{=I} \;\mathbb{E}[X]
	\end{align*} \begin{equation*}
		\boxed{\mathbb{E}[\hat{X}] = \mathbb{E}[X] \underuparrow{=}{X \in \mathbb{R}}X}
	\end{equation*}
	Se ve por lo tanto que el estimador es insesgado. Con este resultado se procede a calcular la covarianza:
	\begin{align*}
		\text{cov}(\hat{X})&= \mathbb{E}\left[(\hat{X}-\mathbb{E}[\hat{X}])(\hat{X}-\mathbb{E}[\hat{X}])^T\right] \\&=\mathbb{E}\left[(\hat{X}-X)(\hat{X}-X)^T\right]
		\\&=\mathbb{E}\left[(A^T A)^{-1} A^T \eta \cdot \eta^T A(A^T A)^{-1}\right]
		\\&= \underbrace{(A^T A)^{-1} A^T \cdot \mathbb{E}\left[\eta\eta^T\right]\cdot A}_{\mathbb{E}\left[\eta\eta^T\right] \, I} (A^T A)^{-1}
	\end{align*} 
\indent Pero como $\mathbb{E}\left[\eta\eta^T\right] = \begin{bmatrix}\sigma^2_{\eta_x}\, I_{2\times2}& 0 \\[0.3em] 0 & \sigma^2_{\eta_y}\, I_{2\times2}\end{bmatrix}$ y $A = \begin{bmatrix} A_1 & 0 \\[0.3em] 0 & A_2\end{bmatrix}$ resulta:
	\begin{equation*} \boxed{\text{cov}(\hat{X}_x) = \sigma^2_{\eta_x}\:(A^T_1 A_1)^{-1}} \end{equation*}
	\begin{equation*} \boxed{\text{cov}(\hat{X}_y) = \sigma^2_{\eta_y}\:(A^T_2 A_2)^{-1}} \end{equation*}

		Realizando la estimación por medio del programa \emph{Octave}, se obtienen los resultados expuestos en la Tabla \ref{tab:ej2} y la ecuación \eqref{eq:cov}.
	\begin{table}[h!]
		\centering
		\begin{tabular}{cccc}
			\toprule
			$\hat{b}_x$	&$\hat{s}_x$		&$\hat{b}_y$	&$\hat{s}_y$\\
			\midrule
			$\num{7.667e-2}$	&$\num{-3.014e-2}$	&$\num{-1.751e-2}$	&$\num{1.004e-2}$\\
			\bottomrule
		\end{tabular}
		\caption{Estimación de los parámetros característicos del error de los acelerómetros en $x$ e $y$.}
		\label{tab:ej2}
	\end{table}

	\begin{equation}
		\text{cov}(\hat{X}) = \begin{bmatrix} \num{1.245e-5}&\num{-8.915e-19}&0&0\\\num{-8.915e-19}&\num{2.603e-7}&0&0\\ 0&0&\num{3.200e-5} & \num{3.265e-10}\\ 0&0&\num{3.265e-10} & \num{6.663e-7}\\\end{bmatrix}
		\label{eq:cov}
	\end{equation}


	\section{Trayectoria del vehículo}\label{sec:tray}
		
	Para realizar el cálculo de la trayectoria (posición del vehículo a lo largo del tiempo) se debe integrar dos veces la aceleración. Sin embargo, como la medición difiere del valor verdadero, al integrarse propaga los errores alterando el valor de la trayectoria.
	\begin{align*}
		{a}_{med}(t) &= {a}_{real}(t) + {\textit{errores}}\,(t)\\
		v_{calc}(t) = v_{inicial} + \int^t_{t_0} a_{med}(\tau)\: d\tau &= v_{real}(t) + \int^t_{t_0} \textit{errores}\,(\tau) d\tau \\
		p_{calc}(t) = p_{inicial} + \int^t_{t_0} v_{calc}(\tau)\: d\tau &= p_{real}(t) + \int^t_{t_0}\left[\int^\tau_{t_0} \textit{errores}\,(\nu) d\nu\right] d\tau %\\
%		\Rightarrow p_{real} = p_{inicial} + \int^t_{t_0} \left[\int^\tau_{t_0} a_{med}\,(\nu) d\nu\right]&+v_{inicial}\; d\tau - \int^t_{t_0}\left[\int^\tau_{t_0} \textit{errores}\,(\nu) d\nu\right] d\tau 
	\end{align*}

	Sin embargo, como toda medición, la trayectoria resultante tendrá incerteza. El error de posición debe ser la mitad de la distancia mínima entre las mediciones. A continuación se desarrolla la relación de dicho error $\delta_p$ con las muestras.
	\begin{equation*}
		\delta_p = \left\|\int^{t_f}_{t_0}\int^{t}_{t_0} (\hat{b}-{b}) + (\hat{s} - s) \cdot a_{med} + \zeta \, dt^{'}\right\| \leq \frac{t^2}{2} |\hat{b}-b| + \underbrace{\int^{tf}_{t_0}\int^t_{t_0}(\hat{s}-s)\cdot a_{med}(t) \, dt}_{\leq\frac{t^2}{2}\cdot|a_{\max}|\cdot|\hat{s}-s|} + \int^{tf}_{t_0}\int^t_{t_0}\zeta(t) \, dt
	\end{equation*}
	Dicha desigualdad puede ser acotada por cotas de sus términos. Los términos correspondientes a la diferencia entre los estimadores y los parámetros reales pueden acotarse suponiendo que las estimaciones son con un $N$ suficientemente grande para que su distribución de probabilidad sea normal. Con dicha hipótesis, se puede afirmar que la mayoría ($\approx 99\%$) de los valores estimados se encuentran a 3 desvios de distancia de la media. Dado que la media de los estimadores es el parámetro a estimar (estimador insesgado), entonces $|\hat{\gamma} - \gamma|\leq 3\,\sigma_\gamma$ donde $\gamma$ es cualquiera de los parámetros estimados.Por otro lado, se puede acotar la integral del ruido por $3\,\sigma_\eta$ debido a que se trata de un proceso de Wiener o de \emph{random walk}. En consecuencia:
	\begin{equation*}
		\delta_p < \frac{t^2_f}{2} (3\sigma_b  + 3|a_{\max}|\sigma_s) + 3\sigma_\eta %< \frac{\min\{\textit{distancia}\}}{2}
	\end{equation*}

	\graficarEPS{0.6}{graf_trayectoria}{Trayectorias con y sin correción, con los posibles puntos de llegada.}{fig:tray}
	A partir de todo lo desarrollado en esta sección, se grafican las trayectorias corregida (y su incerteza) y sin correción en la Figura \ref{fig:tray}. Se puede ver allí, que el entorno de incertidumbre es muy pequeño resultando en que el vehículo no llegó a ninguno, aunque pasó por el \texttt{A}. Éste resultado es extraño y puede deberse a un mal cálculo de los $\delta_p$, porque por inspección se podría afirmar que el vehículo debería haber llegado al punto \texttt{C}.



	\section{Resolución con menos muestras}\label{sec:alternativa}
		
	Suponiendo que por limitaciones en la capacidad de cómputo no se puede resolver el problema de estimación con todas las muestras, se quiere determinar la cantidad mínima para distinguir el punto de llegada del vehículo. Intuitivamente, a menor cantidad de mediciones, mayor es la incerteza en la misma. Por lo tanto se analiza la covarianza de los estimadores. De la Sección \ref{sec:estim} se tiene:
	\begin{equation*}
		cov(\hat{X}) = \begin{bmatrix} \sigma^2_{\eta_x} & 0 \\[0.3em] 0 & \sigma^2_{\eta_y} \end{bmatrix} (A^T A)^{1}
	\end{equation*}

	Se hará el desarrollo para las estimaciones en $x$ pero será equivalente al desarrollo en $y$. Como la covarianza del ruido no varía, se hace foco en la matriz\footnote{donde $\underline{1}=[1 \; 1 \hdots 1]$ y $\underline{\theta}=[\theta_0\; \theta_1 \hdots \theta_n]$}:
	\begin{align*}
		(A^T_1 A_1) &= \begin{bmatrix} \underline{1} \\ -g\cos(\underline{\theta}) \end{bmatrix} \cdot \begin{bmatrix} \underline{1}^T & -g\cos(\underline{\theta}^T) \end{bmatrix}\\
					       &= \begin{bmatrix} N & -g\sum^N_{i=1}\cos(\theta_i) \\[0.7em] -g\sum^N_{i=1} \cos(\theta_i) & g^2\sum^N_{i=1} \cos^2(\theta_i) \end{bmatrix}
	\end{align*}

	Suponiendo que las muestras de $\theta$ está equidistribuidas en el intervalo $[0;2\pi]$, la sumatoria de cosenos es nula. Así se puede ver que la covarianza sigue siendo una matriz diagonal. Entonces resulta:
	\begin{equation*}
	(A^T_1 A_1)^{-1} = \begin{bmatrix} \frac{1}{N} & 0 \\[0.7em] 0 & 1/\left(g^2\sum^N_{i=1} \cos^2(\theta_i)\right) \end{bmatrix}
	\end{equation*}

	Aquí se ve que la cantidad de muestras repercute directamente en la incertidumbre de la estimación. Como se dijo antes, el mismo desarrollo es válido para los estimadores en $y$ diferenciandose en un $\sen$ en vez de un $\cos$. Teniendo en cuenta estos resultados, el error de posición debe ser la mitad de la distancia mínima entre posiciones. Por lo tanto, analizando los puntos \texttt{A, B, C} y \texttt{D} se obtiene:
	\begin{align*}
		d_{\min_x} &= \num{1.620}  &d_{\min_y} = \num{13.26}
	\end{align*}
	
	Por lo tanto, acotando $\sigma_s > g^2 \cdot N$:
	\begin{align*}
		\frac{\min\{distancia\}}{2} &> \delta_p\\
		\rightarrow \min\{distancia\} &> t^2_f (3\, \sigma_b + 3\,|a_{\max}|\,\sigma_s) + 6\, \sigma_{\eta}\\
		\min\{distancia\} &> t^2_f \left(\frac{3}{N} + 3\,\frac{1}{g^2\cdot N}\,|a_{\max}|\right)+ 6\, \sigma_{\eta}\\
		N &> \frac{{t^2_f} \left(3 + 3\,\frac{1}{g^2}\,|a_{\max}|\right)}{\min\{distancia\} - 6\, \sigma_{\eta}}
	\end{align*}
	
	Despejando con los valores de $d_{\min}$ y los datos del problema se obtienen $N_x=2222$ y $N_y=185466$. Éstos valores obtenidos no son los valores mínimos absolutos, sólo representan una cota. Por lo tanto, puede entenderse que haya dado un valor tan alto para $N_y$. 
	Se realiza el gráfico para $N=N_x$ y se expone en la Figura \ref{fig:tray_menos}. Como se ve allí, la aproximación es peor (se aleja del punto \texttt{C}) pero en la coordenada $x$ se encuentra muy próxima como era de esperarse.
	\graficarEPS{0.6}{graf_tray_menos}{Trayectoria del vehículo con $N=2222$.}{fig:tray_menos}

                
	\section{Conclusiones}
		
	Del desarrollo de este trabajo se puede concluir que la estimación de los errores es de suma importancia, porque sin la correción realizada se hubiese confundido la llegada al punto \texttt{B} cuando en realidad se el vehículo en el punto \texttt{C}.\\

	En cuanto a las limitaciones dadas por el poder de cómputo, se garantiza que la cota dada por le incerteza en $x$ se cumple. El problema de la incertidumbre en $y$ se supone que proviene de la proximidad entre los puntos \texttt{A,B} y \texttt{C} en dicho eje. Por lo tanto, para garantizar mejor resolución, se requieren mayor cantidad de muestras.\\

	Se debe tener en cuenta también, que para realizar el cálculo del valor mínimo de $N$ se tomaron muchas cotas máximas. Por lo tanto, dicho valor puede ser mucho menor. Sin embargo, como los errores se deben al instrumental utilizado, no se puede aplicar esta técnica a cualquier problema de trayectoria de un vehículo (los errores pueden diferir dependiendo de la instrumentación).



	
	%\appendix
\end{document}
